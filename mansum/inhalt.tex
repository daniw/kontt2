%coding:utf-8

\section{Ausgangslage}
% Basissatz\\
Im vorliegenden Artikel schreibt Sarah Nigg über einen "`Selbstversuch mit 
Wärmepumpe"'. 

Beat Wellig entwickelte, am Kompetenzzentrum "<Thermische Energiesysteme \& 
Verfahrenstechnik"> der der Hochschule Luzern in Horw, eine 
leistungsgeregelte Luft/Wasser-Wärmepumpe. Diese 
erreicht im Labor eine um bis zu 50 Prozent höhere Effizeinz als herkömmliche 
Luft/Wasser-Wärmepumpen. Das entspricht dem Wirkungsgrad von Wärmepumpen mit 
Erdwärmesonden. 

\section{Vorgehen}
Beim Bau seines Hauses entschied sich Beat Wellig dazu, einen seiner 
Prototypen einzubauen. Für den ungewöhnlichen Testlauf stellte die Firma 
Heliotherm sowohl eine Wärmepumpe als auch eine Frischwasserstation zur 
Verfügung. Im Gegenzug stellt er der Firma die Testergebnisse zur Verfügung. 
Das Luftsystem wurde komplett in Horw entwickelt. Der grösste Aufwand steckt 
dabei im Verdampfer
\footnote{Der Verdampfer entzieht der angesaugten Aussenluft Wärme} 
und in der Leistungsregelung
\footnote{Die Leistungsregelung sorgt dafür, dass die Wärmepumpe nicht mit der
maximalen Leistung läuft. Stattdessen wird die Leistung der Wärmepumpe der 
benötigten Heizleistung angepasst. Dadurch wird der Stromverbrauch der 
Wärmepumpe markant reduziert. }. 
Zurzeit werden über 30 Parameter der Anlage wie Stromverbrauch und 
Heizleistung aufgezeichnet und ausgewertet. 

\section{Ergebnisse}
Der Einbau der leistungsgeregelten Luft/Wasser-Wärmepumpe hatte einige 
Verzögerungen beim Bau seines Hauses zur Folge. So mussten zum Beispiel 
Eingriffe in bereits bestehenden Mauern vorgenommen oder dickere Rohre für 
die Fussbodenheizung verlegt werden. 

Zu Beginn hatte die Anlage noch mit Kinderkrankheiten zu kämpfen. Doch die 
Messungen weisen bisher darauf hin, dass die Wärmepumpe im realen Einsatz 
gleich effizient arbeitet wie im Labor. 

\section{Ausblick}
Bis jetzt interessieren sich noch nicht viele Unternehmen für die 
leistungsgeregelten Luft/Wasser-Wärmepumpen aus Horw. An dieser Stelle könnte 
\textit{polytherm Wärmepumpen} eine Vorreiterrolle spielen. Das würde einen 
Vorsprung auf Konkurenzunternehmen bedeuten und die Innovationskraft des 
Unternehmens unterstreichen. Es wird daher empfohlen, das Kompetenzzentrum 
"<Thermische Energiesysteme \& Verfahrenstechnik"> zu besuchen und die neue 
Technik zu prüfen. 
